\documentclass{beamer}
\usepackage[russian]{babel}
\usepackage[utf8]{inputenc}
\usepackage[T1]{fontenc}
\usepackage{hyperref}
\usepackage{verbatim}
% \usepackage{lmodern}
\usepackage{minted}
\usetheme{Pittsburgh}
\usecolortheme{whale}
\definecolor{links}{HTML}{2A1B81}
\hypersetup{urlcolor=links}
\hypersetup{colorlinks=true,urlcolor=links,linkcolor=links}

\newmintedfile[pyfile]{python}{
fontfamily=tt,
fontsize=\small,
linenos=true,
breaklines=true,
numberblanklines=true,
numbersep=12pt,
numbersep=5pt,
gobble=0,
frame=leftline,
framerule=0.4pt,
framesep=2mm,
funcnamehighlighting=true,
tabsize=4,
obeytabs=false,
mathescape=false
samepage=false, %with this setting you can force the list to appear on the same page
showspaces=false,
showtabs =false,
texcl=false,
}
\newminted[pycode]{python}{
fontfamily=tt,
fontsize=\small,
breaklines=true,
autogobble=true,
frame=lines,
framerule=0.4pt,
funcnamehighlighting=true,
tabsize=2,
}

\title{pytest framework~--- intro \& overview}
\author{Литус Ярослав~-- группа тестирования поиска}
\institute{ООО <<ПП~Спутник>>}


\begin{document}

\begin{frame}
    \titlepage
\end{frame}

\begin{frame}{План}
    \tableofcontents
\end{frame}

\section{unittest vs. pytest}
\begin{frame}[fragile]{Unittest code example}
    \pyfile{./code/unittest-example.py}
\end{frame}

\begin{frame}[fragile]{Same thing done with pytest}
    \pyfile{./code/pytest-example.py}
\end{frame}

\section{assert-выражения и отчет об ошибках}
\begin{frame}[fragile]{Assertions and Error reporting}
    \only<1-2>{
        Unittest:
        \inputminted[frame=lines,linenos,fontsize=\tiny,lastline=11]{bash}{./code/reporting-examples.txt}
        \pause
        Pytest:
        \inputminted[frame=lines,linenos,fontsize=\tiny,firstline=15]{bash}{./code/reporting-examples.txt}
    }
    \begin{itemize}
    \item<3->{Похоже, не правда ли?}
    \item<4->{Но как это достигается?}
    \item<5->{Unittest использует специальные assert-методы класса TestCase.}
    \item<6->{Pytest переопределяет assert-выражения при подгрузке (import) модуля.}
    \uncover<7->{
    \begin{quote}
        pytest has support for showing the values of the most common subexpressions including calls,
        attributes, comparisons, and binary and unary operators.
        (See \href{http://pytest.org/latest/example/reportingdemo.html\#tbreportdemo}{Demo of Python failure reports with pytest}).
        This allows you to use the idiomatic python constructs
        without boilerplate code while not losing introspection information.
    \end{quote}
    }
    \end{itemize}
\end{frame}

\begin{frame}[fragile]{Assert-методы в unittest}
    Самые распространенные assert-методы класса TestCase:
    \begin{pycode}
        assertEqual(a, b)
        assertNotEqual(a, b)
        assertTrue(x)
        assertFalse(x)
        assertIs(a, b)
        assertIsNot(a, b)
        assertIsNone(x)
        assertIsNotNone(x)
        assertIn(a, b)
        assertNotIn(a, b)
        assertIsInstance(a, b)
        assertNotIsInstance(a, b)
    \end{pycode}
\end{frame}

\section{Параметризация}
\begin{frame}[fragile]{Test parametrization}
    \begin{itemize}
    \item Рассмотрим ситуацию когда требуется выполнить один и тот же сценарий с разными данными.
    \pause \item К примеру требуется протестировать строковый метод \verb|isupper()| на разных строках.
    \pause \item Unittest не предоставляет инструмента для параметризации тестов,~--- каждый кейс придется описывать вручную, что приведет к множественному дублированию кода и существенно затруднит его поддержку.
    \pause \item Что же скажет pytest?
    \pause \item Удобная параметризация из коробки!
    \end{itemize}
\end{frame}

\begin{frame}[fragile]{Test parametrization --- example}
    \pyfile{./code/pytest-parametrize.py}
\end{frame}

\section{Выполнение тестов}
\begin{frame}[fragile]{Markers \& test collection}
    \begin{itemize}
    \item Пример запуска тестов:
        \mint{bash}|py.test dir_1 ../tests dir_2/test_this.py|
    \pause \item If no arguments are specified then collection starts from testpaths (if configured) or the current directory. Alternatively, command line arguments can be used in any combination of directories, file names or node ids;
    \pause \item recurse into directories, unless they match \emph{norecursedirs};
    \pause \item \verb|test_*.py| or \verb|*_test.py| files, imported by their test package name;
    \pause \item Test prefixed test classes (without an \verb|__init__| method);
    \pause \item \verb|test_| prefixed test functions or methods are test items.
    \end{itemize}
\end{frame}

\section{Маркеры}
\begin{frame}[fragile]{Markers \& test collection}
    \begin{itemize}
    \item Механизм маркеров дает удобный способ организации тестов и их выборочного запуска.
    \pause \item Как пометить тест(ы) маркером?
    \pause \item 
        \begin{minted}[fontsize=\footnotesize]{python}
        @pytest.mark.smoke
        def test_do_this_and_that():
            # ...

        @pytest.mark.smoke
        @pytest.mark.slow
        def test_do_smth_very_slow():
            # ...
        \end{minted}
    \pause \item Помечая тесты таким образом, мы получаем способ гибко формировать комплекты для запуска:
        \mint[fontsize=\footnotesize]{bash}|py.test tests/ -m smoke|
        \mint[fontsize=\footnotesize]{bash}|py.test tests/ -m "smoke and not slow"|
    \end{itemize}
\end{frame}

\begin{frame}[fragile]{Markers \& test collection}
    \begin{itemize}
    \item Есть несколько полезных встроенных маркеров.
    \pause \item \mint{bash}|py.test --markers|
    \pause \item \mint{python}|@pytest.mark.skipif(condition)|
        Example: \verb|skipif('sys.platform == "win32"')| skips the test if we are on the win32 platform.
        % see \url{http://pytest.org/latest/skipping.html}
    \pause \item \mint{python}|@pytest.mark.xfail(condition, reason, run, raises)|
        See \url{http://pytest.org/latest/skipping.html}
    \pause \item \mint{python}|@pytest.mark.parametrize(argnames, argvalues)|
        Example: \verb|@parametrize('arg1', [1,2])| would lead to two calls of the decorated test function, one with arg1=1 and another with arg1=2
    \end{itemize}
\end{frame}

\section{Немного о фикстурах}
\begin{frame}[fragile]{Немного о фикстурах}
    \begin{itemize}
    \item Одним из самых удачных решений фреймворка pytest стали фикстуры (fixtures).
    \pause \item In software testing, a test fixture is a fixed state of the software
    under test used as a baseline for running tests; also known as the test context.
    \pause \item В классических xUnit-фреймворках фиксуры реализуются через описание действий, относящихся к setUp и к tearDown.
    Например в unittest необходимо переопределить методы setUp() и к tearDown() класса TestCase.
    \pause \item Pytest вводит понятие функции-фикстуры, которая передается в тест в виде funcarg-а.
    \pause \item ЧЕГО???
    \pause \item О фикстурах стоит поговоить отдельно и поподробнее\ldots
    \end{itemize}
\end{frame}

\section{Плагины для pytest}
\begin{frame}[fragile]{Extensions/plugins}
    \begin{itemize}
    \item pytest является во многих смыслах гибким фреймворком.
    \pause \item Подробная документация и богатое API дают возможность достаточно просто писать плагины.
    \pause \item В сообществе уже существует обширный набор: около 200 плагинов
    \pause \item Установка~--- как для обычного пакета: \mint{bash}|pip install pytest-*|
    \pause \item Как подойти к написанию своего плагина?
    \pause \item Все что потребуется~--- изучить \href{http://pytest.org/latest/writing_plugins.html\#pytest-hook-reference}{спецификации предоставляемых хуков}!
    \end{itemize}
\end{frame}

\begin{frame}[fragile]{pytest hooks}
    \only<1>{\inputminted[fontsize=\footnotesize,lastline=17]{python}{./code/pytest-hooks.py}}
    \only<2>{\inputminted[fontsize=\footnotesize,firstline=19]{python}{./code/pytest-hooks.py}}
\end{frame}

\begin{frame}[fragile]{Plugins are easy!}
    Синтетический пример (из другой презентации).
    \pyfile{./code/pytest-plugin-example.py}
    Написанию плагинов стоит посвятить отдельный доклад.
\end{frame}

\section{Полезные ресурсы}
\begin{frame}[fragile]{to be continued\ldots}
    Полезные ресурсы:
    \begin{itemize}
    \item Документация и примеры (Ваш ресурс номер 1): \url{http://pytest.org}.
    Там же коллекция ссылок на доклады и блоги: \url{http://pytest.org/latest/talks.html}
    \item Список плагинов: \url{http://plugincompat.herokuapp.com}
    \item Вопросы-ответы на stackoverflow: \url{http://stackoverflow.com/search?q=pytest}
    \item Интересная перезентация на русском: \url{http://www.slideshare.net/imankulov/pytest-testing}
    \item Интересная перезентация на английском: \url{http://www.the-compiler.org/tmp/sps16.html}
    \item Создатель фреймворка: \url{https://holgerkrekel.net/}
    \end{itemize}
    \footnotesize\emph{При создании данной презентации были использованы материалы некоторых из перечисленных ресурсов}
\end{frame}

\end{document}
