\documentclass{beamer}
\usepackage[russian]{babel}
\usepackage[utf8]{inputenc}
\usepackage[T1]{fontenc}
\usepackage{hyperref}
% \usepackage{lmodern}
\usepackage{minted}
% \usemintedstyle{colorful}
\usetheme{Warsaw}
\hypersetup{colorlinks=true,urlcolor=blue}

\newmintedfile[pyfile]{python}{
fontfamily=tt,
fontsize=\small,
linenos=true,
breaklines=true,
numberblanklines=true,
numbersep=12pt,
numbersep=5pt,
gobble=0,
frame=leftline,
framerule=0.4pt,
framesep=2mm,
funcnamehighlighting=true,
tabsize=4,
obeytabs=false,
mathescape=false
samepage=false, %with this setting you can force the list to appear on the same page
showspaces=false,
showtabs =false,
texcl=false,
}
\newminted[pycode]{python}{
fontfamily=tt,
linenos=true,
breaklines=true,
numberblanklines=true,
numbersep=12pt,
numbersep=5pt,
autogobble=true,
frame=leftline,
framerule=0.4pt,
framesep=2mm,
funcnamehighlighting=true,
tabsize=4,
obeytabs=false,
mathescape=false
samepage=false, %with this setting you can force the list to appear on the same page
showspaces=false,
showtabs =false,
texcl=false,
}

\title{pytest framework~--- intro \& overview}
\author{Литус Ярослав~-- группа тестирования поиска}
\institute{ООО <<ПП~Спутник>>}


\begin{document}

\begin{frame}
    \titlepage
\end{frame}

\begin{frame}{Outline}
    \tableofcontents
\end{frame}




\begin{frame}[fragile]{Unittest code example}
    \pyfile{./code/unittest-example.py}
\end{frame}

\begin{frame}[fragile]{Same thing done with pytest}
    \pyfile{./code/pytest-example.py}
\end{frame}

\begin{frame}[fragile]{Assertions and Error reporting}
    \only<1-2>{
        Unittest:
        \inputminted[frame=lines,linenos,fontsize=\tiny,lastline=11]{bash}{./code/reporting-examples.txt}
        \pause
        Pytest:
        \inputminted[frame=lines,linenos,fontsize=\tiny,firstline=15]{bash}{./code/reporting-examples.txt}
    }
    \begin{itemize}
    \item<3->{Похоже, не правда ли?}
    \item<4->{Но как это достигается?}
    \item<5->{Unittest использует специальные assert-методы класса TestCase.}
    \item<6->{Pytest переопределяет assert-выражения при подгрузке (import) модуля.}
    \uncover<7->{
    \begin{quote}
        pytest has support for showing the values of the most common subexpressions including calls,
        attributes, comparisons, and binary and unary operators.
        (See \href{http://pytest.org/latest/example/reportingdemo.html\#tbreportdemo}{Demo of Python failure reports with pytest}).
        This allows you to use the idiomatic python constructs
        without boilerplate code while not losing introspection information.
    \end{quote}
    }
    \end{itemize}
\end{frame}

\begin{frame}[fragile]{Assert-методы в unittest}
    Самые распространенные assert-методы класса TestCase:
    \begin{pycode}
        assertEqual(a, b)
        assertNotEqual(a, b)
        assertTrue(x)
        assertFalse(x)
        assertIs(a, b)
        assertIsNot(a, b)
        assertIsNone(x)
        assertIsNotNone(x)
        assertIn(a, b)
        assertNotIn(a, b)
        assertIsInstance(a, b)
        assertNotIsInstance(a, b)
    \end{pycode}
\end{frame}

\begin{frame}[fragile]{Test parametrization}
    \begin{itemize}
    \item Рассмотрим ситуацию когда требуется выполнить один и тот же сценарий с разными данными.
    \pause \item К примеру протестировать строковый метод isuuper() на разных строках.
    \pause \item Unittest не предоставляет иструмента для параметризации тестов,~--- для каждый кейс придется описывать вручную, что приведет к множественному дублированию кода и существенно затруднит его поддержку.
    \pause \item Что же скажет pytest?
    \pause \item Удобная параметризация из коробки!
    \end{itemize}
\end{frame}

\begin{frame}[fragile]{Test parametrization --- example}
    \pyfile{./code/pytest-parametrize.py}
\end{frame}

\begin{frame}[fragile]{Немного о фикстурах}
    \begin{itemize}
    \item Одним из самых удачных решений фреймворка pytest стали фикстуры (fixtures).
    \pause \item In software testing, a test fixture is a fixed state of the software
    under test used as a baseline for running tests; also known as the test context.
    \pause \item В xUnit фиксуры реализуются через описание действий, относящихся к setUp и к tearDown.
    Например в unittest необходимо переопределить методы setUp() и к tearDown() класса TestCase.
    \pause \item Pytest вводит понятие функции-фикстуры, которая передается в тест в виде funcarg-а.
    \pause \item ЧЕГО???
    \pause \item О фикстурах стоит поговоить отдельно и поподробнее\ldots
    \end{itemize}
\end{frame}

\begin{frame}[fragile]{Extensions/plugins}
    \begin{itemize}
    \item pytest является во многих смыслах гибким фреймворком.
    \pause \item Подробная документация и богатое API дают возможность достаточно просто писать плагины.
    \pause \item В сообществе уже существует обширный набор: около 200 плагинов
    \pause \item Установка~--- как для обычного пакета: \mint{bash}|pip install pytest-*|
    \pause \item Как подойти к написанию плагина?
    \pause \item Все что потребуется~--- изучить \href{http://pytest.org/latest/writing_plugins.html\#pytest-hook-reference}{спецификации предоставляемых хуков}!
    \end{itemize}
\end{frame}

\begin{frame}[fragile]{pytest hooks}
    \only<1>{\inputminted[fontsize=\footnotesize,lastline=17]{python}{./code/pytest-hooks.py}}
    \only<2>{\inputminted[fontsize=\footnotesize,firstline=19]{python}{./code/pytest-hooks.py}}
\end{frame}

\end{document}
