\documentclass{beamer}
\usepackage[russian]{babel}
\usepackage[utf8]{inputenc}
\usepackage[T1]{fontenc}
\usepackage{hyperref}
\usepackage{verbatim}
\usepackage{minted}
\usetheme{Pittsburgh}
\usecolortheme{whale}
\definecolor{links}{HTML}{2A1B81}
\hypersetup{urlcolor=links}
\hypersetup{colorlinks=true,urlcolor=links,linkcolor=links}

\newmintedfile[pyfile]{python}{
	fontfamily=tt,
	fontsize=\footnotesize,
	linenos=true,
	breaklines=true,
	numberblanklines=true,
	numbersep=12pt,
	numbersep=5pt,
	gobble=0,
	frame=leftline,
	framerule=0.4pt,
	framesep=2mm,
	funcnamehighlighting=true,
	tabsize=4,
	obeytabs=false,
	mathescape=false
	samepage=false,
	showspaces=false,
	showtabs =false,
	texcl=false,
}
\newminted[pycode]{python}{
	fontfamily=tt,
	fontsize=\small,
	breaklines=true,
	linenos=true,
	autogobble=true,
	frame=lines,
	framerule=0.4pt,
	funcnamehighlighting=true,
	tabsize=2,
}



\title{pytest -- фикстуры}
\author{Литус Ярослав / Yaroslav Litus}
\institute{
	Группа тестирования поиска / search testing group \newline
	Поисковый Портал Спутник / \href{http://sputnik.ru}{sputnik.ru}
}
\date{}


\begin{document}

\begin{frame}
	\titlepage
\end{frame}

\begin{frame}{Contents}
	\tableofcontents
\end{frame}

\section{Понятие фикстуры}
\begin{frame}[fragile]{Понятие фикстуры}
	\begin{itemize}
	\item In software testing, a test fixture is a fixed state of the software
	under test used as a baseline for running tests; also known as the test context.
	\pause \item В классических xUnit-фреймворках фикстуры реализуются через описание действий, относящихся к setUp и к tearDown.
	Например в unittest необходимо переопределить методы setUp() и к tearDown() класса TestCase.
	\pause \item pytest вводит понятие функции-фикстуры, которая передается в тестовую функицию/метод в виде funcarg-а (именнованного аргумента).
	\pause \item Фикстуры в pytest реализуют паттерн \href{https://en.wikipedia.org/wiki/Dependency_injection}{Dependency Injection}.
	\end{itemize}
\end{frame}

\section{Фикстуры в pytest}
\subsection{Основы}
\begin{frame}[fragile]{Основы}
	\pyfile{./code/fixtures-basics.py}
\end{frame}

\begin{frame}[fragile]{yield\_fixture}
	\pyfile{./code/fixtures-basics-better.py}
\end{frame}

\subsection{Кэширование}
\begin{frame}[fragile]{Кэширование}
	У декораторов \verb|fixture| и \verb|yield_fixture| есть аргумент \textbf{scope}:
	\pyfile{./code/fixtures-caching.py}
\end{frame}

\begin{frame}[fragile]{Кэширование}
	\inputminted[frame=lines,fontsize=\footnotesize]{bash}{./code/fixtures-caching-out.txt}
	Доступные значения \verb|scope|: \verb|function, class, module, session|
\end{frame}

\subsection{Зависимости между фикстурами}
\begin{frame}[fragile]{Зависимости между фикстурами}
	Фикстуры могут использовать другие фикстуры:
	\pyfile{./code/fixtures-nested.py}
\end{frame}

\subsection{skip/fail в фикстурах}
\begin{frame}[fragile]{skip/fail в фикстурах}
	Вызов \verb|pytest.skip()| или \verb|pytest.fail()| внутри тестовой функции явно пропускает или фэйлит тест.

	То же самое возможно использовать в фикстурах:
	\pyfile{./code/fixtures-skip-fail.py}

	Таким образом все тесты, зависящие от такой фикстуры будут пропущены (skip) или сломаны (fail).
\end{frame}

\subsection{Использование маркеров в фикстурах}
\begin{frame}[fragile]{Использование маркеров в фикстурах}
	Используя доступ из фикстуры к маркерам зависимой функции, возможно реализовать настраиваемую фикстуру:
	\pyfile{./code/fixtures-markers.py}
\end{frame}

\subsection{Autouse-фикстуры}
\begin{frame}[fragile]{Autouse-фикстуры}
	\pyfile{./code/fixtures-autouse.py}
\end{frame}

\begin{frame}[fragile]{Параметризация фикстур}
	\only<1>{
		Фикстуры можно параметризовать. Параметризованные фикстуры можно комбинировать.
		\pyfile{./code/fixtures-parametrize.py}
	}
	\only<2>{
		Применять skip можно и на этапе параметризации:
		\pyfile{./code/fixtures-skipping-params.py}
	}
\end{frame}

\section{Демонстрация}
\begin{frame}[fragile]{Демонстрация}
	\begin{itemize}
	\item Рассмотрим применение фикстур на примере несложных UI тестов.
	\pause \item Сценарий для автоматизации:
	\pause \item Открыть браузер на странице \url{http://sputnik.ru/search}.
	\pause \item Ввести запрос в поле и нажать ENTER.
	\pause \item Сделать необходимую проверку.
	\pause \item Использовать будем:
	\pause \item Pytest (что же еще?),
		\href{http://selenium-python.readthedocs.io/}{Selenium Webdriver} (автоматизация браузера),
		\href{https://github.com/ponty/pyvirtualdisplay}{Virtual Display} (виртуальный дисплей, без иксов),
		\href{http://allure.qatools.ru/}{Allure Framework} (для отчетов).
	\end{itemize}
\end{frame}

\section{Полезные ресурсы}
\begin{frame}[fragile]{Полезные ресурсы}
	\begin{itemize}
	\item ``Advanced py.test fixtures'':
		\url{https://youtu.be/IBC_dxr-4ps} and
		\href{http://devork.be/talks/advanced-fixtures/advfix.html}{slides}
	\item ``Improve your testing with Pytest and Mock'':
		\url{https://youtu.be/RcN26hznmk4}
	\end{itemize}
	% \footnotesize\emph{please don't sue me for copyright infringement}
\end{frame}

\end{document}
