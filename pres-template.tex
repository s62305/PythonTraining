\documentclass{beamer}
\usepackage[russian]{babel}
\usepackage[utf8]{inputenc}
\usepackage[T1]{fontenc}
\usepackage{hyperref}
\usepackage{verbatim}
\usepackage{minted}
\usetheme{Pittsburgh}
\usecolortheme{whale}
\definecolor{links}{HTML}{2A1B81}
\hypersetup{urlcolor=links}
\hypersetup{colorlinks=true,urlcolor=links,linkcolor=links}

\newmintedfile[pyfile]{python}{
    fontfamily=tt,
    fontsize=\small,
    linenos=true,
    breaklines=true,
    numberblanklines=true,
    numbersep=12pt,
    numbersep=5pt,
    gobble=0,
    frame=leftline,
    framerule=0.4pt,
    framesep=2mm,
    funcnamehighlighting=true,
    tabsize=4,
    obeytabs=false,
    mathescape=false
    samepage=false,
    showspaces=false,
    showtabs =false,
    texcl=false,
}
\newminted[pycode]{python}{
    fontfamily=tt,
    fontsize=\small,
    breaklines=true,
    linenos=true,
    autogobble=true,
    frame=lines,
    framerule=0.4pt,
    funcnamehighlighting=true,
    tabsize=2,
}



\title{TITLE}
\author{Литус Ярослав / Yaroslav Litus}
\institute{Группа тестирования поиска / search testing group \newline Поисковый Портал Спутник / \href{http://sputnik.ru}{sputnik.ru}}
\date{}


\begin{document}

\begin{frame}
    \titlepage
\end{frame}

\begin{frame}{Contents}
    \tableofcontents
\end{frame}

\section{section}
\begin{frame}[fragile]{frame}
    \begin{itemize}
        \item You can include python-code directly or from a file.
        \item This example uses \verb|pygments| to colorize the source code.
        \item So keep in mind that the commadline argument \verb|-shell-escape| is required to compile this tex-file.
        \item E.g.: \verb|$ pdflatex -shell-escape pres-template.tex|
        \end{itemize}
\end{frame}

\begin{frame}[fragile]{frame}
    \begin{pycode}
        import antigravity
        print("Yaaaay!")

        def fib(n):
            if n < 0:
                raise ValueError("Must be non-negative")
            elif n == 0 or n == 1:
                return 1
            else:
                return fact(n-2) + fact(n-1)

        print(tuple(fib(i) for i in range(10)))
        assert fib(5) == 8
    \end{pycode}
\end{frame}

\section{section}
\begin{frame}[fragile]{frame}
    \begin{itemize}
    \item item
    \pause \item айтем
    \pause \item Test prefixed test classes (without an \verb|__init__| method);
    \end{itemize}
\end{frame}


\section{section}
\begin{frame}[fragile]{frame}
    \begin{itemize}
    \pause \item 
        \begin{minted}[fontsize=\footnotesize]{python}
        @pytest.mark.smoke
        def test_do_this_and_that():
            # ...

        @pytest.mark.smoke
        @pytest.mark.slow
        def test_do_smth_very_slow():
            # ...
        \end{minted}
    \pause \item Помечая тесты таким образом, мы получаем способ гибко формировать комплекты для запуска:
        \mint[fontsize=\footnotesize]{bash}|py.test tests/ -m smoke|
        \mint[fontsize=\footnotesize]{bash}|py.test tests/ -m "smoke and not slow"|
    \end{itemize}
\end{frame}

\section{Useful resources}
\begin{frame}[fragile]{to be continued\ldots}
    Полезные ресурсы:
    \begin{itemize}
    \item \url{http://pytest.org}
    \item \ldots
    \end{itemize}
    \footnotesize\emph{please don't sue me for copyright infringement}
\end{frame}

\end{document}
